\textbf{Problema 1.} (2.5 puntos) Consideremos la ecuación diferencial
\begin{equation*}
    \frac{d^2 y}{dx^2} - \frac{1}{x}\frac{dy}{dx} + \frac{1-x}{x^2}{y} = 0,
\end{equation*}

dada en el intervalo $x > 0$.

\begin{enumerate}
[label=(\alph*)]
    \item (0.3 puntos) Indicar si el punto $x = 0$ es un punto ordinario o un punto singular de la
    ecuación. En el caso de que sea singular, indicar si es regular o irregular.
    \item (1 puntos) A partir de las raíces de la ecuación indicial de la ecuación diferencial en torno
    al punto $x = 0$, justificar por qué solo una de las dos soluciones linealmente independientes
    de la ecuación diferencial es desarrollable en serie de potencias en torno a $x = 0$. ¿Cuál es
    el radio de convergencia de dicha serie?
    \item (1.2 puntos) Obtener la solución particular, $y = y(x)$, de la ecuación diferencial tal que es
    desarrollable en serie de potencias en torno a $x = 0$ y satisface, además, la condición

    \begin{equation*}
        \lim_{x \rightarrow 0^+} y'(x) = 1
    \end{equation*}

    (debe darse el término general de la serie solución).
\end{enumerate}


\vspace{20px}
\textit{Solución:}
\\

\begin{enumerate}
[label=(\alph*)]
    \item La ecuación diferencial
    es una ecuación lineal homogénea de segundo orden del tipo $y'' + P(x) y' + Q(x) y = 0$ y donde $P(x) = -\frac{1}{x}$
    y $Q(x) = \frac{1-x}{x^2}$. En $x = 0$, ni $P(x)$ ni $Q(x)$ son funciones analíticas, así que $x=0$ no
    puede ser un punto ordinario.\\

    Si ahora se reescribe la ecuación diferencial como

    \begin{equation*}
        y'' + \frac{p(x)}{x} y' + \frac{q(x)}{x^2} y = 0,
    \end{equation*}

    donde $p(x) = x P(x)$ y $p(x) = x^2 Q(x)$, $x = 0$ es un punto singular regular si las funciones $p(x)$ y $q(x)$
    son ambas analíticas en $x = 0$. De otra manera es un punto singular irregular.\\

    Como $p(x) = -1$ y $q(x) = 1 - x$, $x = 0$ es un punto singular regular.

    \vspace{20px}
    \item Siendo $p_0 = p(0)$ y $q_0 = q(0)$, la ecuación de índices de la ecuación diferencial es


    \begin{equation*}
        r(r-1) - r + 1 = 0,
    \end{equation*}

    que tiene una sola raíz doble $r = 1$.

    Entonces, la solución $y(x) = x^r \sum_{n=0}^\infty c_n x^n$
    cuenta sólo con un exponente, y de este modo puede
    haber únicamente una solución en serie de Frobenius.\\

    Ya que $p(x)$ es una constante y $q(x)$ es un polinomio, podemos asegurar que la serie solución va
    a converger (y, por tanto, va a ser válida) para todo $x\;\epsilon\;\mathbb{R}$.


\end{enumerate}
