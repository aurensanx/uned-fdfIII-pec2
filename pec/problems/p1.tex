\textbf{1.}


\vspace{20px}
\textit{Solución:}
\\

\begin{enumerate}
[label=\alph*)]
    \item En el instante inicial, en $t = 0$ s, el número de núcleos de RaF es 0. En ese instante solo tenemos la muestra de $5 \times 10^{-10}$ g de RaE y no
    ha comenzado todavía la desintegración en cadena.

    Esto se puede corroborar sustituyendo $t = 0$ en la expresión del número de núcleos de RaF, de manera que obtenemos $N_F(0) = 0$.

    Por otro lado, la expresión que relaciona el periodo de semidesintegración $t_{1/2}$ con la constante de desintegración $\lambda$ es:

    \begin{equation*}
        t_{1/2} = \frac{\ln 2}{\lambda}
    \end{equation*}

    El valor periodo del semidesintegración es menor para los núcleos de RaE que para los de RaF, de donde deducimos que la constante de desintegración
    $\lambda$ es mayor para los núcleos de RaE:
    $\lambda_{E} > \lambda_{F}$.

    Por ello, del recorrido de la desintegración en cadena será el siguiente:

    Primero, se desintegrarán los núcleos RaE a mayor ritmo que los núcleos RaF por dos razones: en el momento inicial la totalidad de núcleos existentes son
    núcleos RaE y, además, la constante de desintegración es mayor para la desintegración RaE.

    Cuando se hayan desintegrado suficientes núcleos RaE, llegará un momento en el que el ritmo de desintegración de RaF superará al de RaE, al haber mayor número de
    núcleos RaF disponibles, aunque su constante de desintegración sea menor. Este es el tiempo para el cual el número de núcleos RaF será máximo y el tiempo que
    buscamos.

    Tras superar ese punto, el número de núcleos de RaF seguirá disminuyendo hasta que se aproxime a 0 para valores de tiempo muy altos.

    Calcular el tiempo buscado, al que llamamos $t_{m\acute{a}x}$,
    se traduce matemáticamente en derivar la expresión de $N_F (t)$, igualarla a 0 y resolver para $t$. Lo hacemos a continuación.

    \begin{equation*}
        \frac{d N_F (t)}{dt} = \frac{\lambda_E\, N_E^0}{\lambda_F - \lambda_E}
        \Bigl(
        - \lambda_E \, e^{- \lambda_E\, t}  + \lambda_F \, e^{- \lambda_F\, t}
        \Bigr)  = 0
    \end{equation*}

    Simplificando la ecuación anterior llegamos a:

    \begin{equation*}
        \lambda_E \, e^{- \lambda_E\, t} = \lambda_F \, e^{- \lambda_F\, t},
    \end{equation*}

    que despejando $t$ y renombrándola a $t_{m\acute{a}x}$ nos da la expresión:

    \begin{equation*}
        t_{m\acute{a}x} = \frac{1}{ \lambda_E -  \lambda_F }\, \ln{\frac{\lambda_E}{\lambda_F}}
    \end{equation*}

    Operando con los valores de los periodos de semidesintegración dados en el enunciado llegamos a un resultado de $17,21$ días para el
    tiempo en el que el número de núcleos de RaF es máximo.


    \vspace{20px}

    \item Calcular el valor correspondiente para el número máximo de núcleos, $N_{Fm\acute{a}x}$, se reduce a evaluar la expresión de $N_F(t)$
    para el valor $ t_{m\acute{a}x} $ calculado.
    El único valor todavía desconocido de esa expresión es $N_E^0$.

    Sabiendo la masa de la muestra y la masa atómica del elemento ($210 $ u), podemos calcular el valor buscado.

    \begin{equation*}
        N_E^0 = 5 \times 10 ^{-10} \text{ g} \times
        \frac{1 \text{ Kg}}{10^3\text{ g}} \times
        \frac{1 \text{ u}}{1,661 \times 10^{-27}\text{ Kg}} \times
        \frac{1 \text{ núcleo}}{210 \text{ u}} = 1,433 \times 10^{12} \text{ núcleos}
    \end{equation*}

    Con este resultado, calculamos $N_{Fm\acute{a}x}$.

    \begin{equation*}
        N_{Fm\acute{a}x} = N_E^0 \times \frac{\frac{\ln{2}}{5 \text{ días}}}{\frac{\ln{2}}{138 \text{ días}} - \frac{\ln{2}}{5 \text{ días}}} \times
        \Bigl(
        e^{-   \frac{17,21}{5} \ln2} - e^{-\frac{17,21}{138} \ln2}
        \Bigr)  = 1,227 \times 10 ^{12} \text{ núcleos}
    \end{equation*}

    Este resultado es, evidentemente, menor que $N_E^0$.

\end{enumerate}


