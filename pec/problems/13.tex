\question En el espacio vectorial $\mathbb{R}^n$ se consideran los subespacios

\begin{equation*}
    E = \{ (x_1, \ldots, x_n) : \sum\limits_{x=i}\limits^{n} i \cdot x_i = 0 \}, \hspace{8pt} F = \{(t,t, \ldots, t), t \in \mathbb{R}\}
\end{equation*}

\begin{itemize}[$\bullet$]
    \item Calcular sus dimensiones y escribir bases respectivas.
    \item Probar que $\mathbb{R}^n = E \oplus F$.
    \item Escribir explícitamente un vector $(a_1, \ldots, a_n) \in \mathbb{R}^n$ como suma de vectores $e \in E$ y $f \in F$.
    \item Escribir bases de los espacios vectoriales cocientes:
    \begin{equation*}
        \mathbb{R}^n/E,\;\text{y}\;\mathbb{R}^n/F
    \end{equation*}
\end{itemize}

\vspace{20px}
\textit{Solución:}


\begin{itemize}[$\bullet$]

    \item El subespacio vectorial $E$ se puede reescribir como una ecuación implícita:

    \begin{equation*}
        x_1 + 2x_2 + \ldots + nx_n = 0
    \end{equation*}

    $\mathbb{R}^n$ tiene dimensión $n$, y el subespacio $E$ está definido por una ecuación implícita, por lo que $\dim(E) = n - 1$.\\

    Si resolvemos el sistema anterior, calcularemos las ecuaciones paramétricas de $E$, que dependerán de $n-1$ parámetros,
    y de ahí obtendremos una base.\\

    Asignamos valores a las incógnitas secundarias:
    \begin{equation*}
        x_i = \alpha_i\hspace{12pt}  \text{con}\; \alpha_i \in \mathbb{R} \hspace{8pt} \text{y} \hspace{8pt} 2 \leq i \leq n
    \end{equation*}

    Despejamos $x_1$:
    \begin{equation*}
        x_1 = -2\alpha_2 - 3\alpha_3 - \ldots - n\alpha_n
    \end{equation*}

    Expresamos las ecuaciones paramétricas de $E$ como:
    \begin{flalign*}
    (x_1, x_2, x_3, x_4, \ldots, x_{n-1}, x_n)
        = \\
        &\alpha_2(-2, 1, 0, 0, \ldots, 0, 0) + \alpha_3(-3, 0, 1, 0, \ldots, 0, 0) +
        \ldots + \\
        &\alpha_n(-n, 0, 0, 0, \ldots, 0, 1)
    \end{flalign*}

    Para $e_1$ asignamos $a_2 = 1$ y $a_3 = \ldots = a_n = 0$. Operando de manera similar con el resto de parámetros, obtenemos una base de $E$:

    \begin{equation*}
        \{e_1 = (-2, 1, 0, 0, \ldots, 0, 0), e_2 = (-3, 0, 1, 0, \ldots, 0, 0),
        \ldots,
        e_{n-1} = (-n, 0, 0, 0, \ldots, 0, 1)\}
    \end{equation*}

    \vspace{20px}
    Para el subespacio $F$ es trivial decir que $\dim(F) = 1$ y que una base se calcula asignando $t = 1$:
    \begin{equation*}
        \{f_1 = (1, \ldots, 1) \}
    \end{equation*}

    \item $\mathbb{R}^n = E \oplus F$ se cumple si:
    \begin{equation*}
        \mathbb{R}^n = E + F \hspace{12pt} \text{y} \hspace{12pt} E \cap F = \{0\}
    \end{equation*}

    O equivalentemente, si:
    \begin{equation*}
        \dim(E) + \dim(F) = \dim(E+F) = \dim(\mathbb{R}^n ) = n
    \end{equation*}

    Hemos calculado anteriormente que $\dim(E) = n - 1$ y $\dim(F) = 1$, por lo que solo queda demostrar que $\dim(E + F) = n$,
    o lo que es lo mismo,
    que los vectores que componen las bases de los dos subespacios forman un sistema linealmente independiente.\\

    La manera que elegimos para demostrarlo es calculando el determinante de la matriz del sistema por inducción,
    y demostrando que es distinto de cero. La matriz $A_n$ del sistema es:

    \begin{equation*}
        A_n =
        \begin{pmatrix}
            1      & 1      & 1      & \ldots & 1      \\
            -2     & 1      & 0      & \ldots & 0      \\
            -3     & 0      & 1      & \ldots & 0      \\
            \vdots & \vdots & \vdots & \ddots & \vdots \\
            -n     & 0      & 0      & \ldots & 1
        \end{pmatrix}
    \end{equation*}

    Para $n = 2$ tenemos:

    \begin{equation*}
        \det(A_2) =
        \begin{vmatrix}
            1  & 1 \\
            -2 & 1
        \end{vmatrix} = 1 - (-2) = 3
    \end{equation*}

    por lo que el resultado $\det(A_2) \neq 0 $ es cierto. Suponemos que es cierto para $n-1$ y lo demostramos para $n$.

    \begin{equation*}
        \det(A_n) =
        \begin{vmatrix}
            1      & 1      & 1      & \ldots & 1      \\
            -2     & 1      & 0      & \ldots & 0      \\
            -3     & 0      & 1      & \ldots & 0      \\
            \vdots & \vdots & \vdots & \ddots & \vdots \\
            -n     & 0      & 0      & \ldots & 1
        \end{vmatrix}
        = (-1)^{(n+1)}(-n)\det(A_{n1}) + \det(A_{n-1})

    \end{equation*}

    donde hemos hecho uso de la fórmula de Laplace del determinante por la última fila, y donde $A_{n1}$ denota a la submatriz de $A_n$
    que se obtiene eliminando la fila $n$ y la columna $1$.\\

    Desarrollamos este determinante, utilizando en esta ocasión
    la fórmula de Laplace del determinante por la última columna:

    \begin{equation*}
        \det(A_{n1}) =
        \begin{vmatrix}
            1      & 1      & \ldots & 1      & 1      \\
            1      & 0      & \ldots & 0      & 0      \\
            0      & 1      & \ldots & 0      & 0      \\
            \vdots & \vdots & \ddots & \vdots & \vdots \\
            0      & 0      & \ldots & 1      & 0
        \end{vmatrix}
        = (-1)^{(n-1)+1}\det(I_{n-1}) = (-1)^{n}
        
    \end{equation*}

    Sustituyendo el valor calculado en la fórmula anterior:

    \begin{equation*}
        \begin{split}
            \det(A_n)  & = (-1)^{(n+1)}(-n)(-1)^{n} + \det(A_{n-1}) \\
            & = (-1)^{(n+1)}(-1)^{n-1}n + \det(A_{n-1}) \\
            & = (-1)^{2n}n + \det(A_{n-1}) \\
            & = n + \det(A_{n-1})
        \end{split}
    \end{equation*}

    Como $n \in \mathbb{N}$ y $A_{2} = 3$, hemos demostrado que el determinante es distinto de cero para cualquier orden de la matriz
    del sistema lineal, y por tanto, que los vectores de las bases de los subespacios $E$ y $F$ son linealmente independientes.

    \vspace{20px}
    \item Utilizando las bases de los subespacios calculadas en el apartado 1, podemos escribir cualquier vector $a \in \mathbb{R}^n$ como:

    \begin{equation*}
        \begin{split}
        (a_1, a_2, a_3, \ldots, a_n)
            & = f + e \\
            & = \alpha_1 f_1 + (\alpha_2 e_1 + \alpha_3 e_2 + \ldots \alpha_n e_{n-1} ) \\
            & = (\alpha_1, \alpha_1, \alpha_1, \ldots, \alpha_1) +
            (-2\alpha_2-3\alpha_3 - \ldots -n\alpha_n, \alpha_2, \alpha_3, \ldots, \alpha_n)
            \\
        \end{split}
    \end{equation*}


    \vspace{20px}
    \item
    Según lo demostrado en los apartados 1 y 2 de este problema, la base del subespacio vectorial $F$ extiende la base del subespacio vectorial $E$ para formar
    una base de $\mathbb{R}^n$. Por tanto, una base $\mathcal{B}$ de $\mathbb{R}^n/E$ es:

    \begin{equation*}
        \mathcal{B} = \{ f_1 + L(e_1, e_2, \ldots, e_{n-1}) \}

    \end{equation*}

    De manera similar, calculamos una base $\mathcal{B'}$ para el espacio cociente $\mathbb{R}^n/F$:

    \begin{equation*}
        \mathcal{B'} = \{ e_1 + L(f_1) , e_2 + L(f_1) , \ldots, e_{n-1} + L(f_1) \}
    \end{equation*}
\end{itemize}

