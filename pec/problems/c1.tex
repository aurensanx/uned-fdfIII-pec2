\textbf{CUESTIÓN}
\vspace{20px}

El campo eléctrico dentro de un conductor,

\begin{enumerate}[label=\alph*.]
    \item Es siempre cero.
    \item Es cero, excepto en el caso de que haya cargas fijas en algún punto del interior del conductor en cuyo caso
    no es cero.
    \item Es constante, pero no cero.
    \item Es cero, incluso en el caso de que haya cargas fijas en algún punto del interior del conductor.
\end{enumerate}

\vspace{20px}
\textit{Solución:}
\\

Empezamos considerando la situación de un conductor en equilibro electrostático sin cargas fijas en su interior. En ese caso,
el campo eléctrico dentro del conductor debe ser cero. Descartamos por tanto la opción $c$ como respuesta.\\

Consideramos ahora la situación en la que existen cargas fijas en algún punto del interior del conductor. Si aplicamos la ley de Gauss a una superficie
que contenga las cargas fijas y que esté contenida en su totalidad en el interior del conductor,  llegamos a la conclusión de
que el campo eléctrico en esa región no es cero, independientemente de la forma
que tenga esta superficie. Así, podemos descartar las respuestas $a$ y $d$.\\

Las cargas fijas en el interior del conductor inducen una carga de signo opuesto en la superficie interior del conductor.
Las cargas fijas y las inducidas dan como resultado un campo eléctrico nulo fuera de la superficie interior del conductor, pero
no así en la región interior. Por tanto, la respuesta correcta es la $b$.






