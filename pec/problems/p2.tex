\textbf{2.}

\vspace{20px}
\textit{Solución:}
\\

\begin{enumerate}
[label=\alph*)]
    \item Según el conocimiento actual, los leptones son partículas puntuales sin estructura, y se pueden considerar
    elementales en el sentido de que no están compuestas por
    otras partículas. Por lo tanto, la partícula Xi negativa, $\Xi^-$, que según el enunciado se desintegra, no puede ser un leptón.

    Para deducir si es un barión o un mesón numeramos todas las reacciones:

    \begin{align*}
    (1)
        \hspace{20px} & \Xi^- \longrightarrow \Lambda^0 + \pi^- \\
        (2) \hspace{20px} & \Lambda^0 \longrightarrow p + \pi^- \\
        (3) \hspace{20px} & \pi^- \longrightarrow \mu^- + \bar{\nu}_{\mu} \\
        (4) \hspace{20px} & \mu^- \longrightarrow e^- + \nu_{\mu} + \bar{\nu}_{e} \\
    \end{align*}

    En toda reacción o desintegración se debe conservar el número de bariones. Sabemos que el protón es el barión de menor masa.
    En la reacción 2, vemos que un protón es un producto estable de la reacción. Por lo tanto, la única posibilidad es que la partícula
    lambda neutra, $\Lambda^0$, también sea un barión; y por el mismo razonamiento, $\Xi^-$, a su vez, también sea un barión.

    También deducimos que la partícula $\pi^-$, por la conservación del número de leptones, es un mesón.

    \vspace{20px}

    \item Veamos si se conservan los 3 números leptónicos, $L_e$, $L_\mu$ y $L_\tau$
    en cada reacción. Si lo hacen, también se conservarán en el proceso total.

    Según el razonamiento del apartado anterior, en la reacción 1 tenemos a la izquierda un barión ($\Xi^-$)
    y a la derecha un barión ($\Lambda^0$) y un mesón ($\pi^-$), así que los 3 numéricos leptónicos son cero a ambos lados de la reacción y se conservan.

    En la reacción 2 tenemos a la izquierda un barión ($\Lambda^0$) y a la derecha un barión ($p$) y un mesón  ($\pi^-$),
    así que de nuevo los números leptónicos se conservan.

    En la reacción 3 no tenemos ningún leptón a la izquierda de la reacción, pero a la derecha sí.
    La partícula $\mu^-$ es un leptón, y su número leptónico es $L_\mu = +1$.
    La partícula $\bar{\nu}_{\mu}$ es el antineutrino muónico, que al ser un antileptón tiene $L_\mu = -1$.
    Por tanto, la suma total del número leptónico muónico a la derecha de la reacción es 0 y se conserva.

    Por último, en la reacción 4 tenemos a la izquierda de la reacción un muón ($L_\mu = +1$), y a la derecha un electrón
    ($L_e = +1$), un neutrino muónico ($L_\mu = +1$) y un antineutrino electrónico ($L_e = -1$).
    La suma a ambos de la reacción del número leptónico muónico es 1, y del número leptónico electrónico es 0, así que
    ambos se conservan, como en todas las reacciones anteriores, así que podemos asegurar
    que en el proceso total se conservan los tres numéricos leptónicos.


    \vspace{20px}

    \item En todas las reacciones y desintegraciones se conserva el momento angular, pero eso no quiere decir que el espín
    se conserve como en el caso del número de bariones o leptones, ya que el momento angular no solo es causado por el espín de las
    partículas.

    Por tanto, las reacciones no nos aportan información en cuanto al espín de la partícula $\Xi^-$.

    Lo que sí podemos decir es que  $\Xi^-$, al tratarse de un barión, tiene como espines posibles 1/2, 3/2, 5/2, etc.,
    y que por esta razón pertenece a la clase de partículas llamadas fermiones, que obedecen el principio de exclusión de Pauli.


\end{enumerate}

